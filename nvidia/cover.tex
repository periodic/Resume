\documentclass{letter}[11pt]

\addtolength{\topmargin}{-1cm}
\addtolength{\textheight}{3cm}

\signature{Andrew Haven}
\address{Andrew Haven\\
drew.haven@stanford.edu\\
408-982-5425\\
}
\begin{document}

\begin{letter}{NVIDIA Corporation\\
Santa Clara, CA
}

\opening{To whom it may concern:}

I love to program.  It is where I am most comfortable and at home.  My work at Stanford has shown me that the future of programming will be parallel and distributed.  We already have extremely capable GPU processors in our computers, but harnessing them requires developing new languages and looking at new ways of programming. NVIDIA is already tackling these problems.  I want to be a part of that.

It is obvious now that Moore's Law does not hold indefinitely for single-core processors.  We are near the limit of what we can do with a single-core general-purpose processor.  The future of processing is going to be defined by multi-core processors and distributed computing.  NVIDIA already has some of the most powerful processors on the market.  What stands in the way is lack of technology bridging the gap between the programmers and the hardware.  I want to work on filling that gap with new programming languages, compilers, and paradigms.

My vocational background is largely in systems administration and web programming.  However, I have been programming since I was 10 and have never lost the love of burying myself in an afternoon of debugging and I yearn to work with more code on my job.  In order to improve my programming I am currently working towards a Masters in CS at Stanford in my off hours and am constantly working on the side trying out new technologies.  Working while I study allows me to immediately and directly apply the things I learn, and gives me a context for everything I hear in the classroom.

If you have any engineering openings working with compiler development or software related to GPU programming or the Tesla line, I urge you to consider me.  I look forward to working on the next tier of computing power and I hope to hear from you soon.

\closing{Sincerely,}
\end{letter}
\end{document}
